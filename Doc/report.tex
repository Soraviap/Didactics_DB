\documentclass[a4paper,12pt,italian,towside]{article}
\usepackage[latin1]{inputenc}
% Comment the following line to deny the usage of umlauts and other non-ASCII characters
\usepackage[italian]{babel}

%pack per link
\usepackage{hyperref}
\hypersetup{colorlinks=true,linkcolor=blue}

%pack per colori
\usepackage{xcolor}
\usepackage{xcolor,listings}

%pack tabelle
\usepackage{graphicx}



\setcounter{tocdepth}{3}

%titolo e autore
\title{Progettazione Base di Dati}
\author{Francesco Luca Piero Giovanni}
\date{}

%intestazioni e pie pagina
\usepackage{fancyhdr}
\pagestyle{fancy}
\chead{}
\cfoot{\thepage}
\lhead{}
\renewcommand{\headrulewidth}{0.4pt}

%Formattazione per codice SQL
\usepackage{textcomp}
\usepackage{color}

\definecolor{codegreen}{rgb}{0,0.6,0}
\definecolor{codegray}{rgb}{0.5,0.5,0.5}
\definecolor{codepurple}{HTML}{C42043}
\definecolor{codeblue}{HTML}{0000FF}
\definecolor{backcolour}{HTML}{F2F2F2}
\definecolor{bookColor}{cmyk}{0,0,0,0.90}  
\color{bookColor}

\lstset{upquote=true}

\lstdefinestyle{mystyle}{  
	commentstyle=\color{codegreen},
	keywordstyle=\color{codeblue},
	numberstyle=\footnotesize\color{codegray},
	stringstyle=\color{codepurple},
	basicstyle=\footnotesize,
	breakatwhitespace=false,         
	breaklines=true,                 
	captionpos=b,                    
	keepspaces=true,                                   
	numbersep=-10pt,                  
	showspaces=false,                
	showstringspaces=false,
	showtabs=false,      
}
\lstset{style=mystyle} 


%*************************************************************************%
% Start the document
\begin{document}


%generiamo il titolo
\maketitle
\newpage

%indice
\tableofcontents

\newpage
% *****sezione1*****
\section{Descrizione del progetto}
Si vuole realizzare una base di dati per modellare correttamete i Corsi di Caurea dell'Universit\'a di Padova.
Si vogliono rappresentare tutti i dati funzionali ai Corso di Laurea come Scuole, Classi, Curriculum, Attivit\'a formative e relativi SSD, Coorti, Percorsi e Docenti.
In questo progetto non vengono coinvolti studenti o piani di studio.
\par
La modellazione viene fatta coerentemente al sito  \href{https://didattica.unipd.it/}{didattica.unipd.it}, fonte partanto di tutti i requisiti del progetto.



\subsection{Requisiti strutturati}

\textbf{Scuola}
\par Le Scuole hanno funzioni di coordinamento e di razionalizzazione delle attivit\'a didattiche, compresa la proposta di istituzione, attivazione, modifica, disattivazione o soppressione di corsi di laurea, nonch\'e di gestione dei servizi comuni.\textcolor{red!50}{Ogni Scuola pu\'o attivare diversi corsi di laurea in linea con il settore accademico di interesse. Ad ogni Scuola � assegnato un codice a livello Universitario.}
\newline

\textbf{Classe}
\par
\textcolor{red!50}{Le Classi sono dei contenitori che raggruppano i Corsi di Laurea dello stesso ciclo, comunque denominati dagli Atenei, aventi gli stessi obiettivi formativi qualificanti e attivit\'a formative attivate per un numero di crediti e in settori individuati come indispensabili. Le caratteristiche delle Classi sono fissate a livello nazionale, con appositi Decreti Ministeriali e identificate da codici, e sono quindi comuni a tutti gli atenei. I corsi di laurea appartenenti alla stessa Classe hanno identico valore legale.}
\\

\textbf{Corso di Laurea}
\par
\textcolor{red!50} { I Corsi di Laure sono l'insieme di tutte le attivit\'a formative, obbligatorie e non, proposte da un ateneo per conseguire la laurea. La laurea rilasciata \'e coerente al Corso di Laurea scelto. I Corsi di Laurea sono inquadrati in Classi ministeriali, possono prvedere pi� percorsi e afferiscono a una scuola. Per il conseguimento della Laurea, e quindi il completamento del relativo corso, lo studente deve conseguire un numero prefissato di CFU (180 per le lauree triennali). Ogni corso di Laurea appartiene ad un ordinamento legislativo che ne definisce la struttura e la valiidit� legale. Ogni Corso di Laurea pu\'o avere, a driscrezione della Suola, pi� Curricula. Ogni Ateneo identifica un Corso di Laurea con un codice univoco.
} 
\\
\\
\\


\textbf{Curriculum}
\par
\textcolor{red!50} { Un Curriculum � una denominazione di un Corso di Laurea che ne definisce gli obbiettivi oltre che a diffirenziare lo stesso in un insieme di attivit\'a formative differenti. Pi� studenti possono dunque conseguire la stessa larea con percorsi alternativi a seconda dei loro interessi e attitudini scegliendo, ad esempio, un Curriculum Generale piuttosto che un Curriculum Applicativo.
Ogni Ateneo identifica un Curriculum con un codice univoco.
} 
\\


\textbf{Attivit\'a Formativa}
\par
\textcolor{red!50} {Le Attivit\'a Formative sono il cuore di un Corso di Laure e rappresentano tutti gli step che uno studente deve superare per conseguire la Laurea. Le Attivit\'a formative possono essere di quattro tipi: Tirocinio, Lingua, Prova Finale e Insegnamenti. Ogni attivit\'a� formativa si inserisce in un Corso di Laurea in un anno (I, II, II IV o V a seconda del tipo di Laurea) e in un semestre ( I o II). Ogni Attivit\'a Formativa \'e riferita a un Docente responsabile che ha il compito di definirla in termini di: tipo di valutazione, prerequisiti, conoscenze e abilit\'a da acquisire, modalit\'a di esame, criterio di valutazione, contenuti, attivit\'a, materiali, testi consigliati e lingua di emissione. Ogni Attivit\'a formativa si riferisce poi a un Dipartimento some sede fisica delle lezioni. Infine ogni Attivit\'a formativa  legata a uno o pi\'u SSD dai quali, al conseguimento da parte dello studente dell' esame, vengono rilasciati CFU. Le Attivit\'a Formative sono sono soggette a vincoli, decisi dalla Scuola, in termini di superamento di altre Attivit\'a formative precedenti. All'iterno di ogni ateneo le Attivit\'a formative sono identificate da un codice.
} 
\\

\textbf{Docente}
\par
\textcolor{red!50} { I Docenti sono gli attuatori di un Attivit\'a Formativa. Piu Docenti possono essere coinvolti in una stessa Attivit\'a Formativa con ruoli differenti. Dei Docenti coinvolti di un Attivit\'a, uno e uno solo � il Responsabile della stessa e ha il compito di definirla come descritto dal paragrafo Attivit\'a formativa. Dei docenti si vogliono poi poter conoscere tutti i dati relativi ai contatti e agli ambiti di ricerca. Ogni docente pu� partecipare a pi� attivit� formative. Ogni Docente ha una matricola identificativa decisa dall Ateneo.
} 
\\


\newpage
\subsection{Operazioni sulle basi di dati}

\begin{table}[!h] %!h forza la figura sotto il titolo
	\resizebox{\textwidth}{!}{%
		\begin{tabular}{lll}
			\hline
			\textbf{Operazione} & \textbf{Tipo} & \textbf{Frequenza} \\ \hline \hline
			Lista dei prodotti pi`u richiesti & Interrogazione & 4/Mese \\ \hline
			Lista delle materie prime pi`u utilizzate & interrogazione & 1/Mese \\ \hline
			Lista clienti che, pur avendone richiesto almeno uno, non richiedono un ordine da almeno 10 anni & Interrogazione & 1/Anno \\ \hline
			Catalogo di prodotti divisi per categoria con nome e prezzo & Interrogazione & 1/Mese \\ \hline
			Lista ordini non ancora evasi in ordine temporale & Interrogazione & 1/Giorno \\ \hline
			Elenco prodotti con numero di dipendenti dedicati & Interrogazione & 1/Mese \\ \hline
			Lista dipendenti addetti dato un codice prodotto & Interrogazione & 1/Mese \\ \hline
			Aggiornamento Dipendenti & Modifica & 1/Mese \\ \hline
			Aggiornamento Clienti & Modifica & 1/Mese \\ \hline
			Aggiornamento Fornitori & Modifica & 1/Mese \\ \hline
			Aggiornamento Categorie & Modifica & 1/Anno \\ \hline
			Aggiornamento Prodotti & Modifica & 1/Settimana \\ \hline
			Aggiornamento Materie Prime & Modifica & 1/Settimana \\ \hline
			Aggiornamento Ordini & Modifica & 10/Giorno \\ \hline
			Inserimento Forniture & Modifica & 10/Giorno \\ \hline
		\end{tabular}%
	}
\end{table}

\subsection{Glossario}
\begin{table}[!h] %!h forza la figura sotto il titolo
	\resizebox{\textwidth}{!}{%
		\begin{tabular}{llll}
			\hline
			\multicolumn{1}{c}{\textbf{Termine}} & \multicolumn{1}{c}{\textbf{Descrizione}} & \multicolumn{1}{c}{\textbf{Sinonimi}} & \multicolumn{1}{c}{\textbf{Collegamenti}} \\ \hline \hline
			Corso di  Laurea & \begin{tabular}[c]{@{}l@{}}Insieme di tutte le attivita formative ne-\\ cessarie a Laurearsi.\end{tabular} & Corso di studi. & \begin{tabular}[c]{@{}l@{}}Scuola, Classe, \\ Attivit� formativa, \\ Curriculum.\end{tabular} \\ \hline
			\begin{tabular}[c]{@{}l@{}}Istanza attivit� \\ formativa\end{tabular} & \begin{tabular}[c]{@{}l@{}}Insegnamento, Prova finale, Tirocino o \\ Lingua che uno studente trover� effetti-\\ vamente nel suo piando si studi.\end{tabular} & Esame, Corso. & \begin{tabular}[c]{@{}l@{}}Docente, Percorso,\\ Attivit� formativa\end{tabular} \\ \hline
			Responsabile & \begin{tabular}[c]{@{}l@{}}Docente incaricato di istanziare un'at-\\ tivit� formativa e responsabile di tale \\ attivit� per quell anno.\end{tabular} & \multicolumn{1}{c}{-} & \begin{tabular}[c]{@{}l@{}}Atttivit� formativa,\\ Istanza Attivit� \\ formativa, ruolo\end{tabular} \\ \hline
			Coorte & \begin{tabular}[c]{@{}l@{}}Insieme di studenti immatricolatisi nl \\ medesimo anno. Riferendosi a Coorte\\ come anno si identifica il momento di \\ di schedulazione di tutti le Attivit� che\\ gli studenti di quella Coorte vedranno \\ proposte durantela propria carriera.\end{tabular} & Anno, Classe. & \begin{tabular}[c]{@{}l@{}}Attivit� formativa,\\ Percorso, Corso di\\ Laurea.\end{tabular} \\ \hline
			SSD & \begin{tabular}[c]{@{}l@{}}Settore Scentifico Disciplinare:\\ Distinzione disciplinare utilizzata nell\\ universit� in Italia per organizzare l'is-\\ truzione superiore.\end{tabular} & \multicolumn{1}{c}{-} & \begin{tabular}[c]{@{}l@{}}Attivit� formativa,\\ CFU.\end{tabular} \\ \hline
			CFU & \begin{tabular}[c]{@{}l@{}}Credito Formativo Universitario:\\ modalit� utilizzata nelle universit� ita-\\ liane per misurare il carico di lavoro \\ richiesto allo studente.\end{tabular} & Crediti. & \begin{tabular}[c]{@{}l@{}}SSD, Prcorso,\\ Attivit� formativa\end{tabular} \\ \hline
			Percorso & \begin{tabular}[c]{@{}l@{}}Istanza del Corso di Laurea relativa ad\\ un dato Curriculum e a una Coorte.\end{tabular} & Indirizzo. & \begin{tabular}[c]{@{}l@{}}Attivit� formativa,\\ Curriculum Corso \\ di Laurea, Coorte.\end{tabular} \\ \hline
		\end{tabular}%
	}
\end{table}


\newpage

% *****sezione2*****
\section{Progettazione concettuale}

\subsection{Modello concettuale: Entit\`a-Associazione(E-R)}
Di seguito lo schema concettuale prodotto per la rappresentazione della realt`a di interesse:

\begin{figure}[!h] %!h forza la figura sotto il titolo
	\caption{Schema E-R completo}
	\begin{center}
		\includegraphics[scale=0.5,angle=90]{ERdiagram.jpg}
	\end{center}
\end{figure}

\subsection{Dizionario dei dati}

\subsubsection{Entit\`a}

\begin{table}[!h]
\resizebox{\textwidth}{!}{%
	\begin{tabular}{llll}
		\hline
		\textbf{Entit�} & \textbf{Descrizione} & \textbf{Attributi} & \textbf{Identificatore} \\ \hline
		Classe & Raggruppa Corsi di Laurea. & Codice, Descrizione. & Codice. \\ \hline
		SSD & Suddivide le attivit�. & Codice, Descrizione. & Codice. \\ \hline
		Attivit� formativa & Attivit� formativa ancora da istanziare. & Codice, Nome, Tipo insegnamento, CFU. & Codice. \\ \hline
		Coorte & \begin{tabular}[c]{@{}l@{}}Dato un Corso di Laurea offre attivit� \\ Formative e le istanzia.\end{tabular} & Anno. & Anno. \\ \hline
		Curriculum & Specializza Corso di Laurea. & Codice, Nome. & Codice. \\ \hline
		Docente & \begin{tabular}[c]{@{}l@{}}Il responsabile del corso o un docente \\ che ne prende parte.\end{tabular} & \begin{tabular}[c]{@{}l@{}}Matricola, Cognome, Nome, Email, Diparti-\\ mento, Telefono, Qualifica, SSD, Ufficio, \\ Tesi, Aree di Ricerca, Curriculum, Pubblicazioni.\end{tabular} & Matricola. \\ \hline
		Scuola & Attiva Corsi di Laurea. & Codice, Nome. & Codice. \\ \hline
		Corso di Laurea & \begin{tabular}[c]{@{}l@{}}Offre Attivit� Formative, struttura gli \\ indirizzi.\end{tabular} & Codice, Nome,Scuola; Ordinamento & Codice \\ \hline
		Percorso & \begin{tabular}[c]{@{}l@{}}Istanza del Corso di Laurea relativa ad \\ un dato Curriculum e a una Coorte.\end{tabular} & Corso di Laurea, Curriculum, Coorte & \begin{tabular}[c]{@{}l@{}}Corso di Laurea, \\ Curriculum, \\ Coorte\end{tabular} \\ \hline
		\begin{tabular}[c]{@{}l@{}}Istanza Attivit�\\ Formativa\end{tabular} & \begin{tabular}[c]{@{}l@{}}Esame vero e proprio che gli studenti \\ \\ dovranno superare.\end{tabular} & \begin{tabular}[c]{@{}l@{}}Attivit� Formativa, Canale, Anno Accademico,\\  Responsabile, Tipo Valutazione, Dipartimento, \\ a Frequenza Obbligatoria, Corso Singolo, Corso \\ Libero, Prerequisiti, Acquisire, Modalit� Esame, \\ Criteri Valutazione, Contenuti, Attivit�, Materiali, \\ Testi, Lingua.\end{tabular} & \begin{tabular}[c]{@{}l@{}}Attivit� Formativa, Canale, Anno \\ Accademico, Responsabile.\end{tabular} \\ \hline
		&  &  &  \\
		&  &  &  \\
		&  &  &  \\
		&  &  & 
	\end{tabular}%
}
\end{table}

\subsubsection{Associzioni}

Your text goes here...

\subsection{Schema concettuale, regole di vincolo}

\'E necessario definire alcune regole di vincolo per una corretta rappresentazione della realt\'a� di interesse:
\begin{itemize}
	\item RV1: i CFU assegnati ad un corso di laurea devono essere minori della somma dei CFU delle attivit'a formative proposte dai percorsi che realizzano il corso di laurea.
	\item RV2: i CFU totali assegnati ad una attivit\'a formativa devono essere pari alla somma dei CFU assegnati agli SSD che compongono l'attivit\'a formativa.
	\item RV3: la somma dei CFU delle attivit\'a formative offerte per un corso di laurea deve essere maggiore dei CFU richiesti dal corso di laurea per ogni coorte.
	\item RV4: le attivit\'a formative proposte da un percorso per una coorte devono essere offerte dal relativo corso di laurea per la stessa coorte.
	\item RV5: le istanze attivate per un percorso devono essere proposte dallo stesso. 
	\item RV6: le istanze delle attivit\'a formative attivate per un percorso devono svolgersi in un anno accademico uguale o maggiore all'anno della coorte associata al percorso sommato all'anno in cui l'attivit\'a� formativa \'e prevista.
	\item RV7: gli insegnamenti devono avere almeno un SSD e dei crediti.
\end{itemize}

\newpage
% *****sezione3****
\section{Progettazione logica}

\subsection{Ristrutturazione schema E-R}

Your text goes here...

\newpage
\subsection{Modello logico: Relazionale}

\begin{figure} [!h] %!h forza la figura sotto il titolo
	\caption{Schema E-R completo}
	\begin{center}
		\includegraphics[scale=0.455,angle=90]{relationalmodel.png} %file va rinominato senza spazio senno viene plottato il nome file
	\end{center}
\end{figure}

\subsection{Schema logico e regole di vincolo}

\begin{itemize}
	\item RV1: i corsi di laurea devono essere proposti da una scuola.
	\item RV2: i corsi di laurea devono appartenere ad una classe.
	\item RV3: i corsi di laurea devono offrire almeno una attivit\'a formativa.
	\item RV4: i percorsi devono proporre almeno una attivit\'a formativa.
	\item RV5: le istanze delle attivit\'a formative devono essere attivate per almeno un percorso.
	\item RVX: \textcolor{red!50}{SCRIVERE TUTTE LE REGOLE DI VINCOLO DI NON NULLIT\'A}
\end{itemize}

\newpage
% *****sezione4****
\section{SQL}



\subsection{Struttura}

\lstinputlisting[language=SQL]{didactics.sql}

\subsection{Query}

\lstinputlisting[language=SQL]{interrogations.sql}
\newpage
% *****sezione 5*****

\section{Note}

Your text goes here...


% Uncomment the following two lines if you want to have a bibliography. Please do not forget to add an entry to your bibliography and reference it by using the \cite{} command
%\bibliographystyle{alphadin}
%\bibliography{document}

% End of the document
\end{document}